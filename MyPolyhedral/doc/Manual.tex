\documentclass[12pt]{amsart}
\usepackage{amsfonts, amsmath, latexsym, epsfig}
\usepackage{amssymb}
\usepackage{epsf}
\usepackage{url}


\newcommand{\RR}{\ensuremath{\mathbb{R}}}
\newcommand{\NN}{\ensuremath{\mathbb{N}}}
\newcommand{\QQ}{\ensuremath{\mathbb{Q}}}
\newcommand{\CC}{\ensuremath{\mathbb{C}}}
\newcommand{\ZZ}{\ensuremath{\mathbb{Z}}}
\newcommand{\TT}{\ensuremath{\mathbb{T}}}
\newtheorem{proposition}{Proposition}
\newtheorem{theorem}{Theorem}
\newtheorem{corollary}{Corollary}
\newtheorem{lemma}{Lemma}
\newtheorem{problem}{Problem}
\newtheorem{conjecture}{Conjecture}
\newtheorem{claim}{Claim}
\newtheorem{remark}{Remark}
\newtheorem{definition}{Definition}
%\newcommand{\qed}{\hfill $\Box$ }
%\newcommand{\proof}{\noindent{\bf Proof.}\ \ }
\def\QuotS#1#2{\leavevmode\kern-.0em\raise.2ex\hbox{$#1$}\kern-.1em/\kern-.1em\lower.25ex\hbox{$#2$}}


%\usepackage{vmargin}
%\setpapersize{custom}{21cm}{29.7cm}
%\setmarginsrb{1.7cm}{1cm}{1.7cm}{3.5cm}{0pt}{0pt}{0pt}{0pt}
%marge gauche, marge haut, marge droite, marge bas.
\urlstyle{sf}
%\author{Mathieu DUTOUR SIKIRI\'C}

\DeclareMathOperator{\Aut}{Aut}
\DeclareMathOperator{\Sym}{Sym}


\begin{document}

\author{Mathieu Dutour Sikiri\'c}
\address{Mathieu Dutour Sikiri\'c, Rudjer Boskovi\'c Institute, Bijenicka 54, 10000 Zagreb, Croatia, Fax: +385-1-468-0245}
\email{mdsikir@irb.hr}





\title{Partial manual on the polyhedral computer package}


\maketitle

\begin{abstract}
We explain here how to enumerate the extreme Delaunay polytopes
in dimension $n$.
\end{abstract}

\section{Designs principles}

\begin{enumerate}
\item The manual is not supposed to reflect the reality of the code.
The idea is to have the maximal flexibility in the code, so as to maximize
speed and functionality. Manual is for help and is not supposed to reflect
reality.
\item When a function failed to do a computation, it does not return ``fail''.
It stops in the middle with a call to {\rm Print(NullMat(5))} that usually
express the problem.
\item When it is cheap, all sort of checks are made everywhere so as to
ensure security. We want to avoid the code going in wrong directions for
too long. Those checks are for safety, but of course they cannot guarantee
safety. For example, if a group is given for a lattice, then we do check that
the groups, i.e. the generators preserve the lattice, because this is easy.
But we do not check that the full group of the lattice is this one because
this is expensive.
\item All sort of bugs can and do occurr. Nobody is perfect, all papers
and programs contain error. But I will do my best to correct them if at
all possible.
\item The principle is in modularity. Most of the code is in GAP.
When speed is needed use C++ for some auxilliary computations.
\item The goal is, in principle, to be modular. But we have not enforced
with absolute consistency this because i




\end{enumerate}


\end{document}
