\documentclass[12pt]{amsart}
\usepackage{amsfonts, amsmath, latexsym, epsfig}
\usepackage{amssymb}
\usepackage{epsf}
\usepackage{url}
%\usepackage{tikz}


\newcommand{\sfA}{\ensuremath{\mathsf{A}}}
\newcommand{\RR}{\ensuremath{\mathbb{R}}}
\newcommand{\NN}{\ensuremath{\mathbb{N}}}
\newcommand{\QQ}{\ensuremath{\mathbb{Q}}}
\newcommand{\CC}{\ensuremath{\mathbb{C}}}
\newcommand{\ZZ}{\ensuremath{\mathbb{Z}}}
\newcommand{\TT}{\ensuremath{\mathbb{T}}}
\newcommand{\R}{\ensuremath{\mathbb{R}}}
\newcommand{\N}{\ensuremath{\mathbb{N}}}
\newcommand{\Q}{\ensuremath{\mathbb{Q}}}
\newcommand{\C}{\ensuremath{\mathbb{C}}}
\newcommand{\Z}{\ensuremath{\mathbb{Z}}}
\newcommand{\T}{\ensuremath{\mathbb{T}}}
\newtheorem{proposition}{Proposition}
\newtheorem{theorem}{Theorem}
\newtheorem{corollary}{Corollary}
\newtheorem{algorithm}{Algorithm}
\newtheorem{lemma}{Lemma}
\newtheorem{problem}{Problem}
\newtheorem{conjecture}{Conjecture}
\newtheorem{claim}{Claim}
\newtheorem{remark}{Remark}
\newtheorem{definition}{Definition}
\def\QuotS#1#2{\leavevmode\kern-.0em\raise.2ex\hbox{$#1$}\kern-.1em/\kern-.1em\lower.25ex\hbox{$#2$}}


\urlstyle{sf}

\DeclareMathOperator{\Aut}{Aut}
\DeclareMathOperator{\Sym}{Sym}
\DeclareMathOperator{\Isom}{Isom}
\DeclareMathOperator{\vertt}{vert}
\DeclareMathOperator{\conv}{conv}
\DeclareMathOperator{\SC}{SC}
\DeclareMathOperator{\SL}{SL}
\DeclareMathOperator{\GL}{GL}
\DeclareMathOperator{\PSL}{PSL}
\DeclareMathOperator{\Out}{Out}
\DeclareMathOperator{\Min}{Min}
\DeclareMathOperator{\Dom}{Dom}
\DeclareMathOperator{\cone}{cone}
\DeclareMathOperator{\Stab}{Stab}


\begin{document}

\author{Mathieu Dutour Sikiri\'c}
\address{Mathieu Dutour Sikiri\'c, Rudjer Boskovi\'c Institute, Bijenicka 54, 10000 Zagreb, Croatia}
\email{mdsikir@irb.hr}


\title{Manual of the GAP package {\tt cohoperfect}}
\date{}

\maketitle
\tableofcontents

\section{Installation}

A priori the system works only on unix/linux systems.
You need to follow the following steps:
\begin{enumerate}
\item The first step is to install {\tt GAP} which is available at \url{http://www.gap-system.org/}

\item The archive {\bf cohoperfect.tar.gz} can be downloaded from
  {\url{https://sourceforge.net/projects/cohoperfect/?source=directory}}

\item The archive {\bf cohoperfect.tar.gz} should be untarred in the {\bf pkg} directory of GAP.

\item Your File {\bf .gap/gap.ini} must contain the following line:
\begin{verbatim}
SetUserPreference( "InfoPackageLoadingLevel", 4 ); # for additional debugging informations
SetUserPreference( "PackagesToLoad", [ "cohoperfect"]);
\end{verbatim}
if you have no other needed packages. If the file is not existent then you need to create it.

\item Then one needs to run the {\bf ./configure} perl script in the {\bf cohoperfect} directory in order to compile the external programs.
\end{enumerate}
There are only two dependencies to the software:
\begin{enumerate}
\item The GNU gmp multiprecision library. It is installed in many systems and can be obtained via \url{https://gmplib.org/}. On ubuntu the following should work: {\tt sudo apt-get install libgmp-dev}.
\item The latex publishing system, specifically the {\tt latex}, {\tt dvips} and {\tt ps2pdf} programs.  
\end{enumerate}


\section{Computation of the results of the paper}

The computation of the paper can be computed by running the examples in the directory {\tt CohomologyImagQuad/}.
There are two files:
\begin{enumerate}
\item DoEnumerationTest.g which computes just $\GL_3(O_{-3})$.
\item DoEnumerationComplete.g which computes all tables of the paper and takes several days to compute.
\end{enumerate}
Those two scripts are run from the command line by typing
\begin{verbatim}
gap.sh DoEnumerationTest.g
\end{verbatim}
or
\begin{verbatim}
gap.sh DoEnumerationComplete.g
\end{verbatim}
in the directory {\tt CohomologyImagQuad} after the installation.







\end{document}
